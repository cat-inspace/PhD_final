\documentclass[parskip=half]{scrreprt} 
\usepackage[ngerman]{babel} 
\usepackage[latin1]{inputenc}
\usepackage[T1]{fontenc}	
\usepackage{tabularx}
\usepackage{calc}
\usepackage{makecell}
\usepackage{textcomp}
\usepackage[usenames,dvipsnames]{xcolor}


\begin{document}


\begin{table}[]
\caption{Parameters influencing the structure formation of model processed cheese}
\label{tab:lenres}
\begin{tabular}{lll} 
\hline
\textbf{Investigated parameter} &
  \textbf{Key results} \\  \hline
Stirring speed &
  \makecell[l]{Higher processing speed leads to weaker gels \\
 		 high shear might lead to structure corruption} \\  \\
Temperature &
  at least 70\textdegree C necessary to initiate creaming reaction \\  \\
Protein composition &
  \makecell[l]{Model matrix was derived from natural cheese with addition of \\
  		 2\% (w/w protein) protein powder of varying sources . \\
		 Presence of whey proteins, native casein and rennet casein  \\
		 promotes the occurrence of a distinct first exponential phase  \\
		 acid casein and sodium caseinate lead to absence of an early\\
		 exponential increase in viscosity, but show a pronounced  \\
		 exponential increase in apparent viscosity at late processing times} \\  \\
Protein concentration &
  \makecell[l]{Higher concentration in proteins results in stronger gels and \\
  		 stronger display of a step-wise structure build-up} \\  \\
Addition of rework &
  \makecell[l]{values of 5\% and 10\% were investigated \\
  		Highly accelerated structure formation, \\
		increasing with with higher rework concentration}   \\  \\
pH educt &
  Optimum pH for the creaming reaction: 5.83 - 5.96 \\  \\
Fat globule size &
  Smaller Fat globules accelerated structure formation \\  \\
Fat composition &
  \makecell[l]{Use of surface active ingredient in systems prepared w oil \\
   		strongly accelerated structure formation.} \\  \\
Fat concentration & 
\makecell[l]{Lactose was used as dry-matter add on\\
		very low structure formation without presence of fat \\ 
		Presence of fat is needed to display step-wise structure formation} \\  \hline


\end{tabular}

\end{table}

\end{document}